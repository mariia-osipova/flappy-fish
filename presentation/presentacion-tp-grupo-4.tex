%%%%%%%%%%%%%%%%%%%%%%%%%%%%%%%%%%%%%%%%%
% Beamer Presentation
% LaTeX Template
% Version 2.0 (March 8, 2022)
%
% This template originates from:
% https://www.LaTeXTemplates.com
%
% Author:
% Vel (vel@latextemplates.com)
%
% License:
% CC BY-NC-SA 4.0 (https://creativecommons.org/licenses/by-nc-sa/4.0/)
%
%%%%%%%%%%%%%%%%%%%%%%%%%%%%%%%%%%%%%%%%%

%----------------------------------------------------------------------------------------
%	PACKAGES AND OTHER DOCUMENT CONFIGURATIONS
%----------------------------------------------------------------------------------------

\documentclass[
	11pt, % Set the default font size, options include: 8pt, 9pt, 10pt, 11pt, 12pt, 14pt, 17pt, 20pt
	%t, % Uncomment to vertically align all slide content to the top of the slide, rather than the default centered
	%aspectratio=169, % Uncomment to set the aspect ratio to a 16:9 ratio which matches the aspect ratio of 1080p and 4K screens and projectors
]{beamer}

\graphicspath{{Images/}{./}} % Specifies where to look for included images (trailing slash required)

\usepackage{booktabs} % Allows the use of \toprule, \midrule and \bottomrule for better rules in tables

%----------------------------------------------------------------------------------------
%	SELECT LAYOUT THEME
%----------------------------------------------------------------------------------------

% Beamer comes with a number of default layout themes which change the colors and layouts of slides. Below is a list of all themes available, uncomment each in turn to see what they look like.

\usetheme{default}
%\usetheme{AnnArbor}
%\usetheme{Antibes}
%\usetheme{Bergen}
%\usetheme{Berkeley}
%\usetheme{Berlin}
%\usetheme{Boadilla}
%\usetheme{CambridgeUS}
%\usetheme{Copenhagen}
%\usetheme{Darmstadt}
%\usetheme{Dresden}
%\usetheme{Frankfurt}
%\usetheme{Goettingen}
%\usetheme{Hannover}
%\usetheme{Ilmenau}
%\usetheme{JuanLesPins}
%\usetheme{Luebeck}
%\usetheme{Madrid}
%\usetheme{Malmoe}
%\usetheme{Marburg}
%\usetheme{Montpellier}
%\usetheme{PaloAlto}
%\usetheme{Pittsburgh}
%\usetheme{Rochester}
%\usetheme{Singapore}
%\usetheme{Szeged}
%\usetheme{Warsaw}

%----------------------------------------------------------------------------------------
%	SELECT COLOR THEME
%----------------------------------------------------------------------------------------

% Beamer comes with a number of color themes that can be applied to any layout theme to change its colors. Uncomment each of these in turn to see how they change the colors of your selected layout theme.
%
%\usecolortheme{default}
%\usecolortheme{albatross}
%\usecolortheme{beaver}
%\usecolortheme{beetle}
%\usecolortheme{crane}
\usecolortheme{dolphin}
%\usecolortheme{dove}
%\usecolortheme{fly}
%\usecolortheme{lily}
%\usecolortheme{monarca}
%\usecolortheme{seagull}
%\usecolortheme{seahorse}
%\usecolortheme{spruce}
%\usecolortheme{whale}
%\usecolortheme{wolverine}

%----------------------------------------------------------------------------------------
%	SELECT FONT THEME & FONTS
%----------------------------------------------------------------------------------------

% Beamer comes with several font themes to easily change the fonts used in various parts of the presentation. Review the comments beside each one to decide if you would like to use it. Note that additional options can be specified for several of these font themes, consult the beamer documentation for more information.

\usefonttheme{default} % Typeset using the default sans serif font
%\usefonttheme{serif} % Typeset using the default serif font (make sure a sans font isn't being set as the default font if you use this option!)
%\usefonttheme{structurebold} % Typeset important structure text (titles, headlines, footlines, sidebar, etc) in bold
%\usefonttheme{structureitalicserif} % Typeset important structure text (titles, headlines, footlines, sidebar, etc) in italic serif
%\usefonttheme{structuresmallcapsserif} % Typeset important structure text (titles, headlines, footlines, sidebar, etc) in small caps serif

%------------------------------------------------

%\usepackage{mathptmx} % Use the Times font for serif text
\usepackage{palatino} % Use the Palatino font for serif text

%\usepackage{helvet} % Use the Helvetica font for sans serif text
\usepackage[default]{opensans}
\usepackage{hyperref} % Use the Open Sans font for sans serif text
%\usepackage[default]{FiraSans} % Use the Fira Sans font for sans serif text
%\usepackage[default]{lato} % Use the Lato font for sans serif text

\hypersetup{
    colorlinks=true,
    urlcolor=blue,
    linkcolor=blue,
    citecolor=blue,
    pdfborderstyle={/S/U/W 1}
}

%----------------------------------------------------------------------------------------
%	SELECT INNER THEME
%----------------------------------------------------------------------------------------

% Inner themes change the styling of internal slide elements, for example: bullet points, blocks, bibliography entries, title pages, theorems, etc. Uncomment each theme in turn to see what changes it makes to your presentation.

%\useinnertheme{default}
\useinnertheme{circles}
%\useinnertheme{rectangles}
%\useinnertheme{rounded}
%\useinnertheme{inmargin}

%----------------------------------------------------------------------------------------
%	SELECT OUTER THEME
%----------------------------------------------------------------------------------------

% Outer themes change the overall layout of slides, such as: header and footer lines, sidebars and slide titles. Uncomment each theme in turn to see what changes it makes to your presentation.

%\useoutertheme{default}
%\useoutertheme{infolines}
%\useoutertheme{miniframes}
%\useoutertheme{smoothbars}
%\useoutertheme{sidebar}
%\useoutertheme{split}
%\useoutertheme{shadow}
%\useoutertheme{tree}
%\useoutertheme{smoothtree}

%\setbeamertemplate{footline} % Uncomment this line to remove the footer line in all slides
%\setbeamertemplate{footline}[page number] % Uncomment this line to replace the footer line in all slides with a simple slide count

%\setbeamertemplate{navigation symbols}{} % Uncomment this line to remove the navigation symbols from the bottom of all slides

%----------------------------------------------------------------------------------------
%	PRESENTATION INFORMATION
%----------------------------------------------------------------------------------------


\title{Trabajo Práctico Final — Programación Computacional \\[6pt]} % The short title in the optional parameter appears at the bottom of every slide, the full title in the main parameter is only on the title page

\subtitle{Flappy Fish: Juego basado en Pygame} % Presentation subtitle, remove this command if a subtitle isn't required

\author[Osipova, Zanoni, Scofano y Roldan]{Julieta Zanoni, Mariia Osipova, Santino Scofano y Morena Roldan} % Presenter name(s), the optional parameter can contain a shortened version to appear on the bottom of every slide, while the main parameter will appear on the title slide

\institute[UdeSA]{Universidad de San Andrés \\[6pt]
\textit{
jzanoni@udesa.edu.ar \\[1.5pt]
mosipova@udesa.edu.ar \\[1.5pt]
sscofano@udesa.edu.ar \\[1.5pt]
mroldan@udesa.edu.ar}
}

\date[12 de diciembre 2025]{12 de diciembre 2025}


%----------------------------------------------------------------------------------------

\begin{document}

%----------------------------------------------------------------------------------------
%	TITLE SLIDE
%----------------------------------------------------------------------------------------

\begin{frame}
	\titlepage % Output the title slide, automatically created using the text entered in the PRESENTATION INFORMATION block above
\end{frame}

%----------------------------------------------------------------------------------------
%	TABLE OF CONTENTS SLIDE
%----------------------------------------------------------------------------------------

% The table of contents outputs the sections and subsections that appear in your presentation, specified with the standard \section and \subsection commands. You may either display all sections and subsections on one slide with \tableofcontents, or display each section at a time on subsequent slides with \tableofcontents[pausesections]. The latter is useful if you want to step through each section and mention what you will discuss.

\begin{frame}
	\frametitle{Resumen de la presentación}
	\tableofcontents
\end{frame}

%----------------------------------------------------------------------------------------
%	PRESENTATION BODY SLIDES
%----------------------------------------------------------------------------------------

%\section{Text Examples} % Sections are added in order to organize your presentation into discrete blocks, all sections and subsections are automatically output to the table of contents as an overview of the talk but NOT output in the presentation as separate slides

%\section{Introducción}
%
%\section{Arquitectura del Juego}
%
%\section{Algoritmo Genético}
%\subsection{Representación del Individuo}
%\subsection{Función de Aptitud}
%\subsection{Selección, Cruza y Mutación}
%
%\section{Resultados y Visualización}
%
%\section{Conclusiones}

\section{Introducción}

%------------------------------------------------

\begin{frame}
    \frametitle{Introducción}

    \begin{columns}[T]
        \begin{column}{0.35\textwidth}
            Nuestro trabajo práctico está dividido en dos partes: la primera es soble el desarrollo de un videojuego manual inspirado en Flappy Bird, llamado Flappy Fish, implementado en Python con la librería Pygame.
        \end{column}

        \begin{column}{0.65\textwidth}
            \includegraphics[width=\textwidth]{img.png}
        \end{column}
    \end{columns}

\end{frame}

%------------------------------------------------

\begin{frame}
    \frametitle{Introducción}

    \begin{columns}[T]
        \begin{column}{0.35\textwidth}
            La segunda parte se enfoca en la implementación de un Algoritmo Genético (AG) para entrenar a una población de “peces”, se juega de forma autónoma al videojuego manual.        \end{column}

        \begin{column}{0.65\textwidth}
            \includegraphics[width=\textwidth]{img_1.png}
        \end{column}
    \end{columns}

\end{frame}

%------------------------------------------------

\section{Arquitectura del Juego}

%------------------------------------------------
\begin{frame}
    \frametitle{Arquitectura del Juego}

    \begin{columns}[T]
        \begin{column}{0.40\textwidth}
            Pensando en la arquitectura del juego, nos enfrentamos al primer desafío:
            ¿cómo debíamos estructurar y organizar el proyecto?
            Comenzamos trabajando a partir de este borrador inicial.
        \end{column}

        \begin{column}{0.58\textwidth}
            \includegraphics[width=\textwidth]{diagram-initial-class-structure.png}
        \end{column}
    \end{columns}

\end{frame}

%------------------------------------------------

%\begin{frame}
%    \frametitle{Arquitectura del Juego}
%
%    Para entender mejor cómo estructurar el proyecto, analizamos varios juegos desarrollados
%        con Pygame y publicados de forma abierta. Estas referencias nos permitieron observar
%        enfoques comunes de arquitectura y organización del código. Entre ellos, miramos
%        proyectos como \href{https://github.com/mx0c/super-mario-python}{Super Mario Python}
%        y \href{https://github.com/techwithtim/Tower-Defense-Game}{Tower Defence Game},
%        que utilizamos como guía conceptual. \\[5pt]
%
%    \begin{columns}[T]
%        \begin{column}{0.5\textwidth}
%            \includegraphics[width=\textwidth]{img_3.png}
%        \end{column}
%        \begin{column}{0.5\textwidth}
%            \includegraphics[width=\textwidth]{img_2.png}
%        \end{column}
%    \end{columns}
%
%\end{frame}


\begin{frame}
\frametitle{Arquitectura del Juego}

\begin{columns}[T]

    \begin{column}{0.45\textwidth}
        Analizamos varios juegos desarrollados
        con Pygame y publicados de forma abierta. Estas referencias nos permitieron observar
        enfoques comunes de arquitectura y organización del código. Entre ellos, miramos
        proyectos como \href{https://github.com/mx0c/super-mario-python}{Super Mario Python}
        y \href{https://github.com/techwithtim/Tower-Defense-Game}{Tower Defence Game},
        que utilizamos como guía conceptual.
    \end{column}

    \begin{column}{0.52\textwidth}
        \includegraphics[width=\textwidth]{img_3.png}\\[8pt]
        \includegraphics[width=\textwidth]{img_2.png}
    \end{column}

\end{columns}

\end{frame}

%------------------------------------------------

\begin{frame}
    \frametitle{Arquitectura del Juego}

    \begin{columns}[T]
        \begin{column}{1\textwidth}
            \includegraphics[width=\textwidth]{diagram-whole-class-structure.png}
        \end{column}
    \end{columns}

\end{frame}
%------------------------------------------------

\begin{frame}
    \frametitle{Arquitectura del Juego}

    El proyecto se estructura en los siguientes módulos: \\[4pt]

    \begin{itemize}
        \item \textbf{game.py} — clase base del juego (ventana, fondo, sonido y tuberías).
        \item \textbf{fish.py} — física, movimiento y máscara de colisión del pez.
        \item \textbf{screamer.py} — módulo del screamer para el juego.
        \item \textbf{generacion\_de\_tuberias.py} — creación, posición y movimiento de las tuberías.
        \item \textbf{menu.py} — interfaz del menú principal y manejo de opciones.
        \item \textbf{swim\_fish.py} — lógica del juego (manual/modo con Algoritmo Genético).
        \item \textbf{ml/} — política del agente, cálculo del estado, generación de pesos y genética.
    \end{itemize}

\end{frame}

%------------------------------------------------

\subsection{Módulo game.py}

%------------------------------------------------
\begin{frame}[fragile]
\frametitle{Módulo game.py (I)}

La clase \texttt{Game} actúa como el marco general del que hereda \texttt{SwimFish}.
Su constructor realiza la configuración inicial del entorno:

\begin{itemize}
    \item Inicializa Pygame y el objeto \texttt{Clock} (FPS = 120).
    \item Crea la ventana principal del juego (1000×600 píxeles).
    \item Carga los fotogramas del fondo animado.
\end{itemize}

\vspace{4pt}
{\scriptsize
\begin{verbatim}
    class Game:
        def __init__(self):
            pygame.init()
            self.clock = pygame.time.Clock()
            self.FPS = 120
            self.screen = pygame.display.set_mode((1000, 600))

            # Animación de fondo
            self.animation_folder = "../data/img/fondo_animado"
            self.background_frames = self._load_background_frames()
            self.frame_index = 0
            self.frame_rate = 30
\end{verbatim}
}
\end{frame}

%------------------------------------------------

\begin{frame}[fragile]
\frametitle{Módulo game.py (II)}

\begin{itemize}\setlength{\itemsep}{2pt}
    \item Define la música de fondo y sonido del salto.
    \item Sprite de tubería y máscara pixel-perfect.
\end{itemize}

{\scriptsize
\begin{verbatim}
    self.music_path = "../data/audios/linkin park fondo.ogg"
    pygame.mixer.music.load(self.music_path)

    self.sonido_salto = pygame.mixer.Sound(
        "../data/audios/efecto bubble.ogg")

    # Fondo marino
    self.fondo_marino = pygame.image.load(
        "../data/img/pixil-frame-0.png").convert()
    self.fondo_marino = pygame.transform.scale(
        self.fondo_marino, (self.screen_w, self.screen_h))
\end{verbatim}
}
\end{frame}

%------------------------------------------------

\begin{frame}[fragile]
\frametitle{Módulo game.py (III)}

\begin{itemize}\setlength{\itemsep}{2pt}
    \item Define el hueco vertical entre tuberías = \texttt{300}.
    \item Define el evento \texttt{evento\_nueva\_tuberia} cada 1500 ms.
\end{itemize}

{\scriptsize
\begin{verbatim}
    # Tuberías
    self.imagen_tuberia = pygame.image.load(
        "../data/img/alga2.png").convert_alpha()
    self.imagen_tuberia = pygame.transform.scale(
        self.imagen_tuberia, (70, 400))
    self.tuberia_mask = pygame.mask.from_surface(
        self.imagen_tuberia)
    self.hueco_entre_tuberias = 300

    self.evento_nueva_tuberia = pygame.USEREVENT
    pygame.time.set_timer(self.evento_nueva_tuberia, 1500)
\end{verbatim}
}

Entonces, Game sabe todo sobre la ventana, el fondo y las tuberías,
pero no sabe nada sobre quién está volando alrededor de este mundo o cómo se desarrolla exactamente el juego;
esa es responsabilidad de SwimFish.

\end{frame}

%------------------------------------------------

\subsection{Módulo fish.py}

%------------------------------------------------
\begin{frame}[fragile]
\frametitle{Módulo fish.py: Fish y su fisica (I)}

\begin{columns}[T]

    \begin{column}{0.9\textwidth}
        El Fish en fish.py una clase independiente que luego se utiliza dentro de otras clases (principalmente dentro de SwimFish) como un objeto compuesto.
    \end{column}

    \begin{column}{0.1\textwidth}
        \includegraphics[width=\textwidth]{fish1.png}
    \end{column}

\end{columns}

\vspace{6pt}

\begin{itemize}
    \item En el constructor se carga la imagen del pez, se escala al tamaño indicado y se crean su \texttt{rect} y la máscara de colisión.
\end{itemize}

{\scriptsize
\begin{verbatim}
    class Fish:
        def __init__(self, x, y, size, image):

            self._start_pos = (x, y)

            self.size = size
            self.original_image = pygame.image.load(
                image).convert_alpha()
            self.original_image = pygame.transform.scale(
                self.original_image, (size[0], size[1]))
            self.image = self.original_image
            self.rect = self.image.get_rect(center = (x, y))
            self.mask = pygame.mask.from_surface(self.image)
\end{verbatim}
}
\end{frame}

%------------------------------------------------

\begin{frame}[fragile]
\frametitle{Módulo fish.py: Fish y su fisica (II)}

\begin{itemize}
    \item El pez posee una velocidad vertical, una gravedad constante, una fuerza de salto, (donde un valor negativo indica movimiento ascendente) y una velocidad máxima de caída.
    \item El método \texttt{flap()} simplemente reinicia la velocidad al valor de \texttt{jump\_strength}, produciendo un impulso instantáneo hacia arriba.
\end{itemize}

{\scriptsize
\begin{verbatim}
        self.velocity = 0
        self.gravity = 0.3
        self.jump_strength = -10
        self.max_fall_speed = 100
        self.air_resistance = 0.9

    def flap(self):
        self.velocity = self.jump_strength
\end{verbatim}
}
\end{frame}

%------------------------------------------------

\begin{frame}[fragile]
\frametitle{Módulo fish.py: Fish y su fisica (III)}

\begin{itemize}
    \item El método \texttt{update()} incrementa la velocidad con la gravedad, la limita según \texttt{max\_fall\_speed}, actualiza la posición vertical del \texttt{rect} y recalcula la rotación del sprite: el ángulo es proporcional a la velocidad, pero está acotado aproximadamente entre $-30^\circ$ durante el ascenso y $+90^\circ$ durante la caída. Tras la rotación, se recalculan el \texttt{rect} y la máscara.
\end{itemize}

{\scriptsize
\begin{verbatim}
    def update(self):
        self.velocity += self.gravity

        if self.velocity > self.max_fall_speed:
            self.velocity = self.max_fall_speed

        self.rect.y += self.velocity
        self._rotar_pez()

\end{verbatim}
}
\end{frame}

%------------------------------------------------

\begin{frame}[fragile]
\frametitle{Módulo fish.py: Fish y su fisica (IV)}

\begin{itemize}
    \item El método \texttt{\_rotar\_pez()} controla la orientación visual del sprite según su velocidad vertical.
    Calcula un ángulo proporcional a la velocidad: cuando el pez asciende se limita a $-30^\circ$, y cuando cae a $+90^\circ$.
    Rota la imagen original con \texttt{pygame.transform.rotate()} y actualiza el centro del \texttt{rect} para mantener la posición.
    Finalmente, se recalcula la máscara de colisión a partir de la imagen rotada.
\end{itemize}

\vspace{4pt}
{\scriptsize
\begin{verbatim}
    def _rotar_pez(self):
        angulo = self.velocity * 3
        if self.velocity > 0:
            angulo = min(angulo, 90)
        else:
            angulo = max(angulo, -30)
        self.image = pygame.transform.rotate(
            self.original_image, -angulo)
        old_center = self.rect.center
        self.rect = self.image.get_rect(center=old_center)
        self.mask = pygame.mask.from_surface(self.image)
\end{verbatim}
}
\end{frame}

%------------------------------------------------

\begin{frame}[fragile]
\frametitle{Módulo fish.py: Fish y su fisica (V)}

\begin{itemize}
    \item El método \texttt{reset()} restablece el pez a su posición inicial, reinicia su velocidad y reconstruye el \texttt{rect} y la máscara originales. Se usa en el caso de una colision para reiniciar el juego.
\end{itemize}

\vspace{4pt}
{\scriptsize
\begin{verbatim}
    def reset(self):
        self.rect.center = self._start_pos
        self.velocity = 0
        self.image = self.original_image
        self.rect = self.image.get_rect(center=self._start_pos)
        self.mask = pygame.mask.from_surface(self.image)
\end{verbatim}
}
\end{frame}

%------------------------------------------------

\subsection{Módulo generacion\_de\_tuberias.py}

%------------------------------------------------

\begin{frame}[fragile]
\frametitle{Módulo generacion\_de\_tuberias.py (I)}

\begin{columns}[T]

    \begin{column}{0.95\textwidth}
        La clase \texttt{tuberias} en \texttt{generacion\_de\_tuberias.py} define un par de obstáculos: el tubo de algas superior e inferior.
    \end{column}

    \begin{column}{0.05\textwidth}
        \includegraphics[width=\textwidth]{alga2.png}
    \end{column}

\end{columns}

\vspace{6pt}

\begin{itemize}
    \item Se selecciona una \texttt{altura\_referencia} en el rango [150, 450];
    \item A partir de esa posición se construyen los rectángulos de dos tuberias, tal que entre ambos se mantenga un hueco vertical de altura \texttt{hueco} (parámetro recibido desde \texttt{Game}).
\end{itemize}

{\scriptsize
\begin{verbatim}
    class tuberias:
        def __init__(self, x, hueco, imagen):
            self.imagen_tuberia = imagen
            self.altura_referencia = random.randint(150, 450)
            self.x = x
            self.hueco = hueco
            self.tubo_arriba = self.imagen_tuberia.get_rect(
                midbottom = (
                    x, self.altura_referencia - hueco // 2))
            self.tubo_abajo = self.imagen_tuberia.get_rect(
                midtop = (
                    x, self.altura_referencia + hueco // 2))
\end{verbatim}
}
\end{frame}

%------------------------------------------------

\begin{frame}[fragile]
\frametitle{Módulo generacion\_de\_tuberias.py (II)}

\begin{itemize}
     \item El método \texttt{mover\_tuberias()} simplemente desplaza la coordenada \texttt{x} y sincroniza ese movimiento con los rectángulos de ambas tuberías.
    \item El método \texttt{dibujar\_tuberias()} dibuja la tubería superior invertida y la inferior en su orientación normal.
\end{itemize}

{\scriptsize
\begin{verbatim}
    def mover_tuberias(self):
        self.x -= self.velocidad
        self.tubo_arriba.x = self.x
        self.tubo_abajo.x = self.x

    def dibujar_tuberias(self,screen):
        screen.blit(pygame.transform.flip(self.imagen_tuberia, False, True), self.tubo_arriba)
        screen.blit(self.imagen_tuberia, self.tubo_abajo)

\end{verbatim}
}
\end{frame}

%------------------------------------------------

\begin{frame}[fragile]
\frametitle{Módulo generacion\_de\_tuberias.py (III)}

\begin{itemize}
    \item Ambas tuberías comparten una misma coordenada horizontal \texttt{x},
    que posteriormente se reduce a una velocidad constante \texttt{velocidad = 4},
    pidiendo que las tuberias se desplacen de derecha a izquierda.
    \item Se almacena \texttt{gap\_y = altura\_referencia}, es decir,
    la coordenada vertical del centro del pasaje,
    utilizada para calcular la característica \texttt{dy} del agente.
\end{itemize}

{\scriptsize
\begin{verbatim}
    self.velocidad = 4
    self.gap_y = self.altura_referencia
\end{verbatim}
}
\end{frame}


%------------------------------------------------
\subsection{Módulo menu.py}
%------------------------------------------------

\begin{frame}
\frametitle{Módulo menu.py (I)}

El menú en \texttt{menu.py} recibe la pantalla y sus dimensiones, y crea el título \texttt{FLAPPY FISH!} junto con dos opciones: o el juego manual o la simulacion (AG).

\vspace{6pt}

\centering
    \includegraphics[width=0.8\textwidth]{img_4.png}

\end{frame}

%------------------------------------------------
\begin{frame}[fragile]
\frametitle{Módulo menu.py (II)}

El método \texttt{manejar\_eventos()} devuelve el modo de juego cuando el usuario hace clik con el mouse sobre la opción correspondiente o al presionar las teclas 1/2.

{\scriptsize
\begin{verbatim}
    def manejar_eventos(self, event):
            if event.type == pygame.MOUSEBUTTONDOWN:
                if event.button == 1:
                    mouse_pos = event.pos
                    if self.rect_single.collidepoint(mouse_pos):
                        self.seleccion = 'SINGLE'
                        return self.seleccion
                    if self.rect_evolutivo.collidepoint(mouse_pos):
                        self.seleccion = 'EVOLUTIVO'
                        return self.seleccion

            if event.type == pygame.KEYDOWN:
                if event.key == pygame.K_1:
                    self.seleccion = 'SINGLE'
                    return self.seleccion
                elif event.key == pygame.K_2:
                    self.seleccion = 'EVOLUTIVO'
                    return self.seleccion
\end{verbatim}
}
\end{frame}


%------------------------------------------------

\subsection{Module main.py}

%------------------------------------------------

\begin{frame}[fragile]
\frametitle{Módulo main.py (I)}

El método \texttt{manejar\_eventos()} devuelve el modo de juego cuando el usuario hace clik con el mouse sobre la opción correspondiente o al presionar las teclas 1/2.

{\scriptsize
\begin{verbatim}
    def manejar_eventos(self, event):
            if event.type == pygame.MOUSEBUTTONDOWN:
                if event.button == 1:
                    mouse_pos = event.pos
                    if self.rect_single.collidepoint(mouse_pos):
                        self.seleccion = 'SINGLE'
                        return self.seleccion
                    if self.rect_evolutivo.collidepoint(mouse_pos):
                        self.seleccion = 'EVOLUTIVO'
                        return self.seleccion

            if event.type == pygame.KEYDOWN:
                if event.key == pygame.K_1:
                    self.seleccion = 'SINGLE'
                    return self.seleccion
                elif event.key == pygame.K_2:
                    self.seleccion = 'EVOLUTIVO'
                    return self.seleccion
\end{verbatim}
}
\end{frame}
%------------------------------------------------

\begin{frame}
\frametitle{LALALALALLALA}


\end{frame}

%\section{Modo manual del Juego}

\end{document}


