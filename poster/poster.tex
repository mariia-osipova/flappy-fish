% Unofficial University of Cambridge Poster Template
% https://github.com/andiac/gemini-cam
% a fork of https://github.com/anishathalye/gemini
% also refer to https://github.com/k4rtik/uchicago-poster

\documentclass[final]{beamer}

% ====================
% Packages
% ====================

\usepackage{tabularx}
\usepackage[T1]{fontenc}
\usepackage{lmodern}
\usepackage[orientation=portrait,size=a2,scale=1.15]{beamerposter}
\usetheme{gemini}
\usecolortheme{nott}
\usepackage{graphicx}
\usepackage{booktabs}
\usepackage{tikz}
\usepackage{pgfplots}
\pgfplotsset{compat=1.14}
\usepackage{anyfontsize}

% ====================
% Lengths
% ====================

% If you have N columns, choose \sepwidth and \colwidth such that
% (N+1)*\sepwidth + N*\colwidth = \paperwidth
\newlength{\sepwidth}
\newlength{\colwidth}
\setlength{\sepwidth}{0.025\paperwidth}
\setlength{\colwidth}{0.45\paperwidth}

\newcommand{\separatorcolumn}{\begin{column}{\sepwidth}\end{column}}

% ====================
% Title
% ====================
\definecolor{HeaderFg}{HTML}{0F2438}

% НЕ задаём bg для headline, иначе тема закрасит шапку поверх текстуры
\setbeamercolor{headline}{bg=, fg=HeaderFg}
\setbeamercolor{title in headline}{bg=, fg=HeaderFg}
\setbeamercolor{author in headline}{bg=, fg=HeaderFg}
\setbeamercolor{institute in headline}{bg=, fg=HeaderFg}

% --- СДЕЛАТЬ ШАПКУ КОРОЧЕ (это ключевой параметр) ---
\newlength{\headerbgheight}
\setlength{\headerbgheight}{0.15\paperheight} % было 0.22 -> стало короче


\newlength{\footerbgheight}
\setlength{\footerbgheight}{0.11\paperheight}

\setbeamercolor{footline}{bg=, fg=HeaderFg}
\setbeamercolor{footer}{bg=, fg=HeaderFg}

% (рекомендую) чуть уменьшить элементы в шапке, чтобы всё влезло
\setbeamerfont{title in headline}{size=\LARGE}
\setbeamerfont{author in headline}{size=\large}
\setbeamerfont{institute in headline}{size=\normalsize}

% и/или уменьшите логотипы:
% \logoleft{\includegraphics[height=2.6cm]{logos/fish.png}}
% \logoright{\includegraphics[height=2.6cm]{logos/university-logo.pdf}}

% --- Фон: белая страница + текстура только в шапке ---
\usebackgroundtemplate{%
\begin{tikzpicture}[remember picture,overlay]

  % 1) фон всей страницы с запасом за края (убирает hairlines)
  \fill[white]
    ([xshift=-1mm,yshift=-1mm]current page.south west)
    rectangle
    ([xshift= 1mm,yshift= 1mm]current page.north east);

  % 2) фон только в области шапки
  \begin{scope}
  \clip (current page.north west)
    rectangle ([yshift=-\headerbgheight]current page.north east);

  \fill[black!6]
    (current page.north west) rectangle ([yshift=-1mm]current page.north east);

  \draw[line width=1pt] (current page.north west)
  ++(0,-\headerbgheight) -- ++(\paperwidth,0);

  % картинка чуть выше по высоте, без yshift
  \node[
  opacity=0.25,
  anchor=north,
  inner sep=0pt,
  outer sep=0pt
  ] at (current page.north) {%
    \includegraphics[
      width=\paperwidth,
      height=\dimexpr\headerbgheight+2pt\relax,
      trim=0 0 0 2pt,
      clip
    ]{logos/background.png}
  };

  \fill[white,opacity=0.15] (current page.north west)
    rectangle ([yshift=-\headerbgheight]current page.north east);
  \end{scope}

  % 3) финальный “колпачок” по самому верху (на случай артефакта рендера)
  \fill[black!6]
    ([yshift=1mm]current page.north west) rectangle (current page.north east);

    % 2b) фон только в области НИЖНЕЙ шапки (footer)
  \begin{scope}
    \clip (current page.south west)
      rectangle ([yshift=\footerbgheight]current page.south east);

    % подложка (с небольшим запасом за края — против hairlines)
    \fill[black!6]
      ([xshift=-1mm,yshift=-1mm]current page.south west)
      rectangle
      ([xshift= 1mm,yshift=\footerbgheight]current page.south east);

    % текстура (привязка к низу)
    \node[
      opacity=0.25,
      anchor=south,
      inner sep=0pt,
      outer sep=0pt
    ] at (current page.south) {%
      \includegraphics[
        width=\paperwidth,
        height=\dimexpr\footerbgheight+2pt\relax,
        trim=0 2pt 0 0, % если у background.png есть прозрачная кромка снизу
        clip
      ]{logos/background.png}%
    };

    % лёгкая "вуаль" как в шапке
    \fill[white,opacity=0.15]
      (current page.south west) rectangle ([yshift=\footerbgheight]current page.south east);

    % линия по ВЕРХУ футера
    \draw[line width=1pt]
      ([yshift=\footerbgheight]current page.south west) --
      ([yshift=\footerbgheight]current page.south east);
  \end{scope}
    % 2b) фон только в области НИЖНЕЙ шапки (footer)
  \begin{scope}
    \clip (current page.south west)
      rectangle ([yshift=\footerbgheight]current page.south east);

    % подложка (с небольшим запасом за края — против hairlines)
    \fill[black!6]
      ([xshift=-1mm,yshift=-1mm]current page.south west)
      rectangle
      ([xshift= 1mm,yshift=\footerbgheight]current page.south east);

    % текстура (привязка к низу)
    \node[
      opacity=0.25,
      anchor=south,
      inner sep=0pt,
      outer sep=0pt
    ] at (current page.south) {%
      \includegraphics[
        width=\paperwidth,
        height=\dimexpr\footerbgheight+2pt\relax,
        trim=0 2pt 0 0, % если у background.png есть прозрачная кромка снизу
        clip
      ]{logos/background.png}%
    };

    % лёгкая "вуаль" как в шапке
    \fill[white,opacity=0.15]
      (current page.south west) rectangle ([yshift=\footerbgheight]current page.south east);

    % линия по ВЕРХУ футера
    \draw[line width=1pt]
      ([yshift=\footerbgheight]current page.south west) --
      ([yshift=\footerbgheight]current page.south east);
  \end{scope}

\end{tikzpicture}%
}

\title{Flappy Fish: Juego basado en Pygame}


\author{%
  \vspace{19pt}
  Julieta Zanoni \and Mariia Osipova \\
  Santino Scofano \and Morena Roldan
}

\institute[shortinst]{Univercidad de San Andrés}

% ====================
% Footer (optional)
% ====================

\footercontent{%
  \vspace*{0.1\footerbgheight}%
  \footnotesize
  % левая колонка
  \begin{minipage}[c]{0.40\textwidth}
    \raggedright
    Universidad de San Andrés \\[4pt]
    \textit{%
      jzanoni@udesa.edu.ar \\[1.5pt]
      mosipova@udesa.edu.ar \\[1.5pt]
      sscofano@udesa.edu.ar \\[1.5pt]
      mroldan@udesa.edu.ar}
  \end{minipage}%
  \hfill
  % Центр: опускаем текст вниз, выравнивание по центру
  \begin{minipage}[c]{0.1\textwidth}
    \includegraphics[height=0.6\footerbgheight]{logos/fish-big.png}\\[2pt]
  \end{minipage}%
  \begin{minipage}[c]{0.1\textwidth}
    \includegraphics[height=0.6\footerbgheight]{logos/fish-big.png}\\[2pt]
  \end{minipage}%
  \begin{minipage}[c]{0.1\textwidth}
    \includegraphics[height=0.6\footerbgheight]{logos/fish-big.png}\\[2pt]
  \end{minipage}%
  % правая колонка
  \begin{minipage}[c]{0.20\textwidth}
    \raggedleft
    \includegraphics[height=0.6\footerbgheight]{logos/qr.png}\\[2pt]
    \footnotesize
    ¡Este QR te dirigirá al repositorio\\
    de nuestro proyecto!
  \end{minipage}%
}


% (can be left out to remove footer)


% ====================
% Logo (optional)
% ====================

%\logoright{%
%  \hspace*{-2cm}% <-- чем больше по модулю, тем левее
%  \includegraphics[height=1.5cm]{logos/fish-game.png}%
%}


\logoright{}

\addtobeamertemplate{headline}{}{%
\begin{tikzpicture}[remember picture,overlay]
  \node[anchor=north east, xshift=-2.5cm, yshift=-3.5cm]
    at (current page.north east)
    {%
      \includegraphics[height=1.8cm]{logos/img.png}%
    };
\end{tikzpicture}%
}


\newlength{\fishX}
\setlength{\fishX}{15ex}

% --- насколько водоросль "заходит" под рыбку:
%  0pt  = ровно встык слева
%  >0   = чуть под рыбку (обычно выглядит лучше)
%  <0   = небольшой зазор между ними
\newlength{\algaOverlap}
\setlength{\algaOverlap}{1cm}

% --- водоросль: слева от рыбки, от нижнего края шапки ---
\addtobeamertemplate{headline}{% ДО содержимого шапки (будет под рыбкой/текстом)
  \begin{tikzpicture}[remember picture,overlay]
    \begin{scope}
      % ограничиваем областью шапки
      \clip (current page.north west)
        rectangle ([yshift=-\headerbgheight]current page.north east);

      % нижний левый угол шапки
      \node[
        anchor=south east,
        inner sep=0pt,
        outer sep=0pt,
        xshift=\dimexpr\fishX+\algaOverlap\relax,
        yshift=1pt
      ] at ([yshift=-\headerbgheight]current page.north west) {%
        \includegraphics[height=8cm]{logos/alga2.png}%
      };
    \end{scope}
  \end{tikzpicture}%
}{}

% --- рыбка слева в логотипе ---
\logoleft{%
  \hspace{\fishX}%
  \includegraphics[height=7cm]{logos/fish-big.png}%
}%


% (если у тебя есть логотип справа через \logoright, оставь его как был)
% например:
% \logoright{\includegraphics[height=1.8cm]{logos/img.png}}

% --- водоросль, прижатая к НИЗУ шапки, рисуем оверлеем ---
% предполагается, что \headerbgheight уже задана, например:
% \setlength{\headerbgheight}{3cm}
%
%\addtobeamertemplate{headline}{}{%
%  \begin{tikzpicture}[remember picture,overlay]
%    % точка: левый нижний угол шапки (top - \headerbgheight)
%    \node[
%      anchor=south west,
%      xshift=8ex % сдвиг вправо от левого края; подстрой по вкусу
%    ] at ([yshift=-\headerbgheight]current page.north west)
%    {%
%      \includegraphics[height=7cm]{logos/alga2.png}%
%    };
%  \end{tikzpicture}%
%}



% ====================
% Body
% ====================

\begin{document}

% Refer to https://github.com/k4rtik/uchicago-poster
% logo: https://www.cam.ac.uk/brand-resources/about-the-logo/logo-downloads
% \addtobeamertemplate{headline}{}
% {
%     \begin{tikzpicture}[remember picture,overlay]
%       \node [anchor=north west, inner sep=3cm] at ([xshift=-2.5cm,yshift=1.75cm]current page.north west)
%       {\includegraphics[height=7cm]{logos/unott-logo.eps}};
%     \end{tikzpicture}
% }

\begin{frame}[t]
\begin{columns}[t]
\separatorcolumn

\begin{column}{\colwidth}

  \begin{block}{Resumen del proyecto}

    Lo primero, que se ve, cuando se ejecuta el codigo es el menu, donde se puede elegir el modo del juego.

    \begin{figure}
      \centering
      % ----- верхний ряд -----
      \begin{minipage}[t]{0.48\textwidth}
        \centering
        \includegraphics[width=\linewidth]{logos/menu.png}
        \caption{Menú principal}
      \end{minipage}\hfill
      \begin{minipage}[t]{0.48\textwidth}
        \centering
        \includegraphics[width=\linewidth]{logos/fish-game.png}
        \caption{Flappy Fish en el modo manual}
      \end{minipage}

      \vspace{0.7cm} % расстояние между рядами

      % ----- нижний ряд -----
      \begin{minipage}[t]{0.48\textwidth}
        \centering
        \includegraphics[width=\linewidth]{logos/modoag.png} % <- картинка с modo automático, если есть
        \caption{Flappy Fish en el modo automático}
      \end{minipage}\hfill
      \begin{minipage}[t]{0.48\textwidth}
        \centering
        \includegraphics[width=\linewidth]{logos/findeljuego.png} % <- сюда картинку со смертью рыбы
        \caption{Estado de game over (pez muerto)}
      \end{minipage}

    \end{figure}

    Nuestro trabajo práctico se divide en dos partes: primero, el desarrollo
    de un videojuego tipo Flappy Bird llamado Flappy Fish, implementado en Python con Pygame;
    y segundo, la implementación de un Algoritmo Genético para entrenar una población
    de peces que juegan de forma autónoma.

  \end{block}

  \vspace{25pt}

  \begin{alertblock}{Objetivos del Proyecto}

    El proyecto va más allá de la implementación básica de la simulación,
    centrándose en el diseño de una plataforma operativa dual y el análisis
    del rendimiento algorítmico.


    \begin{itemize}
      \item \textbf{Implementar una Arquitectura Modular de Dualidad Operativa:} Diseñar y construir una arquitectura completa y
      modular capaz de integrar dos modos de juego distintos (Manual y Algorítmico). El objetivo es demostrar la separación
      de responsabilidades (lógica de juego vs. lógica algorítmica), garantizando la funcionalidad completa y
      optimizada de cada modo dentro de una misma base de código.
      \vspace{17pt}
      \item \textbf{Optimizar el Comportamiento de Agentes mediante Aprendizaje Automático:} Optimizar el Comportamiento de Agentes mediante Aprendizaje Automático:
      \vspace{17pt}
      \item \textbf{Visualizar y Analizar la Evolución de la Aptitud Algorítmica:} Desarrollar herramientas de visualización gráfica para monitorear y analizar la evolución del proceso de optimización. Esto incluye graficar la evolución de la aptitud de los agentes a lo largo de las generaciones y evaluar la influencia de los parámetros de política y los pesos en el rendimiento del modelo.

    \end{itemize}

  \end{alertblock}

  \vspace{17pt}

  \begin{block}{Metodología y Dualidad Operativa}

    Este proyecto de arquitectura completa se diseñó para explorar
    y contrastar dos paradigmas fundamentales: la interacción manual directa con el usuario
    y la automatización algorítmica mediante técnicas de aprendizaje automático.
    La implementación se realizó bajo un diseño modular que garantiza la dualidad operativa,
    permitiendo la integración fluida de ambos modos dentro de un marco unificado. El proyecto cuenta con dos modos de operación distintos, cada uno con un enfoque y objetivos diferentes:
    1) Modo Manual (Single Player): jugabilidad y experiencia del usuario; y 2) Modo Algorítmico (Simulación): simulación y apredizaje automático.

%    \vspace{17pt}
%
%    \begin{tabularx}{\linewidth}{@{}X X@{}}  % два столбца одинаковой ширины
%
%      \textbf{1. Modo Manual (Single Player)} &
%      \textbf{2. Modo Algorítmico (Simulación)} \\[0.5em]
%
%      \vspace{10pt}
%
%      \textbf{Enfoque principal:} jugabilidad y experiencia del usuario. &
%      \textbf{Enfoque principal:} simulación y aprendizaje automático. \\[0.4em]
%
%      \textbf{Mecanismo de control:} control directo por el usuario (teclado). &
%      \textbf{Mecanismo de control:} algoritmo evolutivo (estrategia autónoma). \\[0.4em]
%
%      \textbf{Detalle operativo:} el bucle principal del juego lee eventos de teclado
%      (\texttt{ESPACIO}), llamando directamente a la función \texttt{fish.flap()}. Se priorizan
%      la interfaz, el audio y la experiencia interactiva. &
%      \textbf{Detalle operativo:} se genera una población de agentes donde cada uno posee
%      una política de salto propia, definida por un vector de pesos que parametriza su
%      toma de decisiones. No se utiliza interacción manual. \\
%
%    \end{tabularx}
  \end{block}

\end{column}

\separatorcolumn

\begin{column}{\colwidth}

  \begin{block}{Arquitectura general del juego}

    La arquitectura del proyecto se diseñó de manera modular y orientada a objetos.
    Este enfoque permitió establecer una estructura clara y organizada que facilita
    la separación de responsabilidades: la lógica general del juego,
    la física del agente (pez), la gestión de elementos visuales (tuberías, screamer) y,
    fundamentalmente, la lógica de la automatización algorítmica (AG).

    \begin{figure}
      \centering
      \includegraphics[width=\linewidth]{logos/arq.png}\\[15pt]
      \small Figure 5: Flowchart de la arquitectura del proyecto
    \end{figure}

\end{block}

%  \begin{block}{Estructura y Módulos Principales}
%
%    La implementación se organiza en una serie de módulos de Python,
%    con una clase principal (Game) que gestiona el contexto general y coordina las funcionalidades específicas.
%    Los principales módulos y su rol son:

%    \begin{itemize}
%      \item \texttt{game.py}: Contiene la clase base (\texttt{Game}) para la inicialización
%            y el contexto general del juego (ventana, fondo, sonido y la gestión de tuberías).
%      \item \texttt{menu.py}: Gestiona la interfaz del menú principal y el manejo de opciones
%            al inicio del juego.
%      \item \texttt{fish.py}: Implementa la física, el movimiento (función \texttt{flap()})
%            y la máscara de colisión del pez para el modo manual.
%      \item \texttt{generacion\_de\_tuberias.py}: Se encarga de la creación, posición
%            y el movimiento de los obstáculos (tuberías).
%      \item \texttt{screamer.py}: Módulo dedicado a la gestión de elementos visuales especiales,
%            como el jumpscare.
%      \item \texttt{ml/vector\_w.py} — generación de vectores de pesos iniciales.
%      \item \texttt{ml/policy.py} — definición de la política \texttt{decidir(dy,dx,vy)}.
%      \item \texttt{ml/calcular\_estado.py} — construcción del estado $(dy,dx)$ a partir de la próxima tubería.
%      \item \texttt{ml/genetics.py} — selección proporcional, cruza, mutación y nueva generación.
%    \end{itemize}
%  \end{block}

  \begin{exampleblock}{Diseño Algorítmico y Fundamento Matemático del AG}

    El núcleo de la simulación reside en una política de salto
    $\gamma : \mathbb{R}^{3} \to \{\text{True}, \text{False}\}$,
    que determina si el pez debe saltar basándose en el estado del juego:
    la diferencia vertical $\mathrm{dy}$, la distancia horizontal $\mathrm{dx}$ a la próxima tubería
    y la velocidad vertical $v_y$.

    La política se modela como una combinación lineal del estado normalizado,
    ponderada por un vector de pesos
    \[
      \mathbf{w} = [w_0, w_1, \dots, w_5],
    \]
    de modo que el pez salta si una función lineal del estado supera cierto umbral
    (por ejemplo, si $z > 0$).

    La clave del Algoritmo Genético (AG) es la selección proporcional
    (\emph{``método de la ruleta''}), que utiliza el \emph{fitness}
    (distancia recorrida y el tiempo de supervivencia) de cada individuo.

%    La probabilidad de que el individuo $i$  sea elegido como padre es:
%
%    \[
%      P(i) = \frac{\max(f_i)^2}{\sum_j \max(f_j)^2}.
%    \]

    \begin{minipage}[t]{0.48\textwidth}
      \vspace{0pt}%
      \raggedright

      \begin{itemize}
        \item Imaginamos una ruleta donde cada individuo ocupa un sector.
        \item El tamaño de cada sector es proporcional a su fitness.
        \item Giramos la ruleta (elegimos un número aleatorio) y vemos en qué sector cae.
        \item El individuo de ese sector es el padre seleccionado.
      \end{itemize}

      La probabilidad de que el individuo $i$ sea elegido como padre es:

      \[
        P(i) = \frac{\max(f_i, 0)^2}{\sum_j \max(f_j, 0)^2}.
      \]
    \end{minipage}%
    \hfill
    \begin{minipage}[t]{0.48\textwidth}
      \vspace{0pt}%
      \centering
      \includegraphics[width=\linewidth]{logos/CHART.png}\\[0.5em]
      \small Figure 6: Visualizacion de la ruleta
  \end{minipage}


  \end{exampleblock}

  \begin{block}{Conclusión}

    En esencia, el proyecto materializa con éxito una arquitectura íntegra y modular, demostrando la viabilidad de integrar dos paradigmas operativos: la interacción manual del usuario y la simulación autónoma. Además, sirvió como una plataforma de prueba robusta para la experimentación con algoritmos, la validación de técnicas evolutivas en la toma de decisiones y un laboratorio de simulación de Machine Learning.

  \end{block}

\end{column}

%  \begin{block}
%    \centering
%    IA Fest UdeSA  Diciembre, 2025
%  \end{block}

\separatorcolumn

\end{columns}
\end{frame}

\end{document}
